
\documentclass[t]{beamer}
\usetheme[department=winuk,official=false,theme=blue,titlebgimage=pumas.png,innovation=true]{tue2008}
\usepackage[english]{babel}
\usepackage{listings,amsmath}
\usepackage{minted}
%\usepackage[utf8]{inputenc}


\graphicspath{{images/}}

% Load syntax highlighting for LaTeX programming code in this presentation
\lstset{language=TeX,
        basicstyle=\color{black}\ttfamily,
        commentstyle=\color{gray}\it\ttfamily,
        keywordstyle=\color{tuered}\bf\ttfamily,
        showstringspaces=false,
        frame=single,
        backgroundcolor=\color{white},
        moretexcs={usetheme,frametitle,setbeamercovered,setbeameroption,usebackgroundtemplate,movie,logo,note,uncover,chapter,subsection,subsubsection,EUR,EURofc,includegraphics,lstset,color,it,bf,RequirePackage,pause,overlay,frontmatter,backmatter,mainmatter,maketitle,setlength,fancyhf,fancyhead,fancyfoot,lhead,chead,rhead,lfoot,cfoot,rfoot,texteuro,textcelsius,appendix,selectlanguage,part,tableofcontents}
}

% Define a description list where you can set the width of the widest label
\newenvironment{descrsf}[1]
  {\begin{list}{}{\renewcommand{\makelabel}[1]{\textsf{##1}\hfil}
                  \setlength{\itemsep}{0.5em}
                  \setlength{\parsep}{0pt}
                  \settowidth{\labelwidth}{\textsf{#1}}
                  \setlength{\labelsep}{10pt}
                  \setlength{\leftmargin}{\labelwidth}
                  \addtolength{\leftmargin}{\labelsep}
                  \providecommand{\descriptionlabel}[1]%
                      {\hspace{\labelsep}\textsf{#1}}
                 }
  }
  {\end{list}}

\title{Publication and Management System (PUMAS) \\ CSI 605}
\author{Ofentse Jabari, Boipelo Mosetho, Ontiretse Ishmael}


 \begin{document}

\begin{titleframe}
% You can insert additional text for the title frame here if you want
\end{titleframe}

%\setbeamercovered{transparent=30}
%\begin{frame}
%	\frametitle{Outline}
	%\tableofcontents[pausesections]
%\end{frame}
%\setbeamercovered{invisible}



\section{Introduction}
\begin{frame}
	\frametitle{Introduction}
	
\end{frame}





\section{Plagiarism Detection}
\begin{slidetop}
	\frametitle{Main Objective}
	
	Below are the fully implemented and extensively tested modules of PUMAS
	
	\begin{enumerate}
		\item Document Uploading.
		\item Document search (with different search criterias i.e. title, authors, department, type)
		\item Downloading of documents.
		\item Plagiarism checker
		\item User Authentication
		
	\end{enumerate}

\end{slidetop}




\section{Plagiarism Detection}
\begin{frame}
		\frametitle{Modules/Packages Used}
		
		
		\inputminted[firstline=1, 
					 lastline=5, 
		             frame=lines,
		             fontsize=\footnotesize
		             ]{python}{../pumas/docxToText.py}

\end{frame}

\section{Plagiarism Detection}
\begin{frame}
	\frametitle{Converting document to plain text}
	
	
	\inputminted[firstline=14, 
	lastline=17, 
	frame=lines,
	fontsize=\footnotesize
	]{python}{../pumas/docxToText.py}
	
	(Excluding pictures, tables or any other graphical components.)
	
	\inputminted[firstline=19, 
	lastline=25, 
	frame=lines,
	fontsize=\footnotesize
	]{python}{../pumas/docxToText.py}
	
	
\end{frame}


\section{Plagiarism Detection}
\begin{frame}
	\frametitle{Converting document to plain text}
	
	Example {\tiny (before removing empty lines)}:\\
	
	['01. Introduction', ' ', 'this chapter is all about related literature review ...', 'the literature review covers,
	 the role of music...', ...]
	 
	 \inputminted[firstline=27, 
	 lastline=37, 
	 frame=lines,
	 fontsize=\footnotesize
	 ]{python}{../pumas/docxToText.py}
	
	
\end{frame}




\section{Plagiarism Detection}
\begin{frame}
	\frametitle{Creating sentence tokens}
	
	Create sentence lexicon of tokenized (and have variant forms of the same word) sentences.
	
	\inputminted[firstline=79, 
	lastline=94, 
	frame=lines,
	fontsize=\footnotesize
	]{python}{../pumas/docxToText.py}	
	
\end{frame}



\section{Plagiarism Detection}
\begin{frame}
	\frametitle{Comparing two sentences}
	
	sentence = ['chapter', 'related', 'literature', 'review', 'extent', 'music', 'promote', 'social']\\
	sentence2 = ['music', 'chapter', 'field', 'extend', 'associated', 'literature', '.']\\
	
	
	sentence\_match = [1, 1, 0, 1, 0, 1, 0, 0]
	
	\inputminted[firstline=127, 
	lastline=131, 
	frame=lines,
	fontsize=\footnotesize
	]{python}{../pumas/docxToText.py}
	
	sent\_perc\_matches = [45, 10, 0, 1.5, 0, 0, 0, 30]	
	
\end{frame}



\section{Plagiarism Detection}
\begin{frame}
	\frametitle{Comparing two sentences}
	
	scores = [45, 50, 40, 15, 70, 55, 10, 30]
	
	\inputminted[firstline=145, 
	lastline=151, 
	frame=lines,
	fontsize=\footnotesize
	]{python}{../pumas/docxToText.py}	
	
\end{frame}






\end{document} 